\documentclass[11pt,a4paper]{report}
\usepackage[utf8]{inputenc}
\usepackage[french]{babel}
\usepackage[T1]{fontenc}
\usepackage{amsmath}
\usepackage{amsfonts}
\usepackage{amssymb}
\usepackage[left=2cm,right=2cm,top=2cm,bottom=2cm]{geometry}
\begin{document}
\tableofcontents
\chapter{Le phénomène de diffraction}
	\section{Rappels théorique}
	\section{Schéma de principe}
	\section{Liste du matériel}
	\begin{itemize}
	\item laser monochromatique
	\item mètre ruban
	\item dias à fente rectangulaire
	\item dias à fente circulaire
	\item porte dia
	\item écran de projection
	\item statif
	\end{itemize}
	\section{Principe de l'expérience}
	\section{Tableau de mesures}
	\subsection{Expérience avec fente rectangulaire}
	\begin{tabular}{|c|c|c|c|c|c|c|}
		\hline
		\bf n & \bf D & \bf $\Delta$D & \bf z & \bf $\Delta$z & \bf $\lambda$& \bf $\Delta\lambda$ \\
		\hline
		 & \bf [mm] & \bf [mm] & \bf [mm] & \bf [mm] & \bf [nm]  & \bf [nm] \\
		\hline
		4 & 5670 & 50 &  &  1 &&  \\
		5 & 5670 & 50 &  &  1 && \\
		\hline
	\end{tabular}
	\subsection{Expérience avec fente circulaire}
	\begin{tabular}{|c|c|c|c|c|c|c|}
		\hline
		\bf n & \bf D & \bf $\Delta$D & \bf z & \bf $\Delta$z & \bf $\oslash$& \bf $\Delta\oslash$ \\
		\hline
		 & \bf [mm] & \bf [mm] & \bf [mm] & \bf [mm] & \bf [nm]  & \bf [nm] \\
		\hline
		4 & 5670 & 50 & 34&  1 &&  \\
		5 & 5670 & 50 & 42&  1 && \\
		\hline
	\end{tabular}
	\section{Calculs}
		\subsection{Expérience avec fente rectangulaire}
		\subsubsection{Calcul de la longueur d'onde $\lambda$ du rayon laser}
		\begin{equation}
		\lambda = \frac{a.z_{n}}{n.D}		
		\end{equation}		 
		où
		 \begin{itemize}
		  \item a est la largeur de la fente
		  \item D est la distance écran-dia
		  \item n est ordre
		  \item z est distance du minima 
		 \end{itemize}
		
				
		\subsubsection{Calcul de l'incertitude de $\lambda$}
		\begin{equation}
		\frac{\Delta \lambda}{\lambda} = \frac{\Delta\left(\frac{a.z_{n}}{n.D}\right)}{\frac{a.z_{n}}{n.D}} 
		= \frac{\Delta D}{D}+\frac{\Delta z_{n}}{z_{n}}
		\end{equation}
		\begin{equation}
		\Delta \lambda
		= \left(\frac{\Delta D}{D}+\frac{\Delta z_{n}}{z_{n}}\right).\lambda
		\end{equation}
		\begin{center}
		$\Delta b = \left(\frac{50}{5670}+\frac{,,}{,,}\right).,, = $
		\end{center}
		\subsection{Expérience avec fente circulaire}
		\subsubsection{Calcul de la longueur d'onde $\lambda$ du rayon laser}
		\subsubsection{Calcul de l'incertitude de $\lambda$}
	\section{Conclusion}
\chapter{Le phénomène d'interférence}
	\section{Rappels théorique}
	\section{Schéma de principe}
	\section{Liste du matériel}
	\begin{itemize}
	\item laser monochromatique
	\item mètre ruban
	\item dias à paires de fentes
	\item porte dia
	\item écran de projection
	\item statif
	\end{itemize}
	\section{Principe de l'expérience}
	\section{Tableau de mesures}
	\begin{tabular}{|c|c|c|c|c|c|c|c|}
		\hline
		\bf n & \bf D & \bf $\Delta$D & \bf z & \bf $\Delta$z & \bf b & \bf $\Delta$b & \bf $\alpha$ \\
		\hline
		 & \bf [mm] & \bf [mm] & \bf [mm] & \bf [mm] & \bf [mm]  & \bf [mm]  &  \\
		\hline
		1 & 5670 & 50 & 3 &   1&0,59&  & $\frac{\pi}{2}$\\
		2 & 5670 & 50 & 10 &  1&0,53&  & $\frac{3\pi}{2}$\\
		3 & 5670 & 50 & 16 &  1&0,55&  & $\frac{5\pi}{2}$\\
		4 & 5670 & 50 & 22 &  1&0,56&  & $\frac{7\pi}{2}$\\
		5 & 5670 & 50 & 28 &  1&0,57&  & $\frac{9\pi}{2}$\\
		6 & 5670 & 50 & 32 &  1&0,61&  & $\frac{11\pi}{2}$\\
		7 & 5670 & 50 & 38 &  1&0,60&  & $\frac{13\pi}{2}$\\
		8 & 5670 & 50 & 44 &  1&0,60 &  & $\frac{15\pi}{2}$\\
		\hline
	\end{tabular}
	\section{Calculs}
		Calcul réalisé pour la dernière ligne du tableau
		\subsection{Calcul de l'écart entre les 2 fentes b}
		\begin{equation}
		\alpha = \frac{\pi.b.z_{n}}{\lambda.D} \rightarrow b = \frac{\alpha .\lambda.D }{\pi.z_{n}} 
		\end{equation}
\begin{center}$ b = \frac{\frac{15\pi}{2}.622,5.10^{-6}.5670}{\pi.44} = 0,60 mm$\end{center}
		\subsection{Calcul de l'incertitude de b}
		\begin{equation}
		\frac{\Delta b}{b} = \frac{\Delta\left(\frac{\alpha .\lambda.D }{\pi.z_{n}}\right)}{\frac{\alpha .\lambda.D }{\pi.z_{n}}} 
		= \frac{\Delta D}{D}+\frac{\Delta z_{n}}{z_{n}} + \frac{\Delta \lambda}{\lambda}
		\end{equation}
		\begin{equation}
		\Delta b
		= \left(\frac{\Delta D}{D}+\frac{\Delta z_{n}}{z_{n}} + \frac{\Delta \lambda}{\lambda}\right).b
		\end{equation}
		\begin{center}
		$\Delta b = \left(\frac{50}{5670}+\frac{1}{3} + \frac{\Delta \lambda}{622,5.10^{-6}}\right).0,60 = $
		\end{center}
	\section{Conclusion}
\chapter{Les réseaux de diffraction}
	\section{Rappels théorique}
	\section{Schéma de principe}
	\section{Liste du matériel}
	\section{Principe de l'expérience}
	\section{Tableau de mesures}
	\section{Calculs}
		\subsection{Calcul de ...}
		\subsection{Calcul de l'incertitude de ...}
	\section{Conclusion}
\end{document}